\documentclass[a4paper]{article}
\usepackage[slovene]{babel}
\usepackage[utf8]{inputenc}
\usepackage[T1]{fontenc}
\usepackage{marvosym}
\usepackage{amssymb,amsmath}
\title{Razvrščanje z dominantnimi množicami - povzetek}
\author{Taja Debeljak, Anže Marinko \\ Finančni praktikum \\ Finančna matematika, Fakulteta za matematiko in fiziko}
\date{Jesen 2017}

\newtheorem{definition}{Definicija}[section]
\newtheorem{theorem}{Izrek}

\begin{document}
\maketitle
\section{Uvod}
Grupiranje (ang. Clustering) je postopek razvrščanja predmetov znotraj razreda v podrazrede (cluster) tako, da so si predmeti znotraj istega podrazreda bolj podobni med sabo, kot so si podobni z elementi iz ostalih podrazredov. \\
Problem združevanja lahko opišemo z uteženim grafom, ki ga definiramo kot trojico $G = (V,E,\omega)$, kjer je $V = {1,\ldots,n}$ končna množica vozlišč, $E \subseteq V \times V$ množica usmerjenih
povezav in $\omega: E \rightarrow \mathbb{R}$ funkcija, ki vsakemu vozlišču dodeli neko vrednost(težo). Vozlišča grafa G ustrezajo predmetom, ki jih je potrebno združevati. \\
Povezave predstavljajo, kateri predmeti so med seboj povezani, utežene povezave pa odražajo podobnosti med povezanimi predmeti. Poleg tega matrika $A_{i,j} = \omega(i,j) \text{ za vse } i, j \in V$ predstavlja podobnost med vozlišči. Imenujemo jo matrika podobnosti. \\
Osnovni lastnosti, ki morata zadostovati gruči, sta:
\begin{itemize}
\item Notranja homogenost: elementi, ki pripadajo gruči si morajo biti med seboj podobni
\item Maksimalnost: gruče ne moremo dodatno razširiti z uvedbo zunanjih elementov
\end{itemize}

\begin{definition}
Naj graf G predstavlja primer združevanja množic in naj bo $C \subseteq V$ neprazna podmnožica. \textit{Povprečna utežena vhodna stopnja} glede na C je definirana kot
$$awindeg_C(i) = \frac{1}{\lvert C \rvert}\sum_{j\in C}A_{i,j}$$
kjer $\lvert C \rvert$ predstavlja velikost množice C. Za $j\in C$ definiramo
$$\phi_C(i,j) = A_{i,j} - awindeg_C(j)$$
Funkcija $\phi_C(i,j)$ je \textit{mera relativne podobnosti} elementa i z elementom j glede na povprečno povezanost elementa i z elementi iz C.\\
\textit{Težo elementa} i glede na množico C definiramo kot
$$W_C(i)=\begin{cases}
1& \text{; če $\lvert C \rvert = 1$},\\
\sum_{j\in C\setminus{i}}\phi_{C\{i\}}(i,j)W_{C\setminus\{i\}}(j)& \text{; sicer}.
\end{cases}$$
Vrednost $W_C(i)$ nam pove koliko podpore prejme element i od elementov $C\setminus\{i\}$ glede na skupno podobnost z elementi iz $C\setminus\{i\}$. Pozitivne vrednosti nam povedo da je i močno
koleriran z $C\setminus\{i\}$.\\
\textit{Skupna teža množice} C pa je definirana z
$$W(C) = \sum_{i\in C}W_C(i)$$
\end{definition}
\begin{definition}{Dominantna množica}\\
Neprazni množici $C \subseteq V$ za katero je $W(T) > 0$ za vsako neprazno množico $T \subseteq C$ pravimo dominantna množica, če velja:
\begin{enumerate}
\item $W_C(i) > 0$ za vse $i \in C$
\item $W_{C\cup\{i\}}(i) < 0$ za vse $i \notin C$
\end{enumerate}
\end{definition}

\section{Povezava s teorijo optimizacije}
Če se omejimo na simetrične povezanosti, torej A je simetrična matrika, potem lahko dominantno množico zapišemo kot rešitev naslednjega standardnega kvadratičnega programa
\begin{gather}
max f(x) = x^TAx \\
\text{p. p. } x\in\Delta \subset \mathbb{R}^n
\end{gather}
Kjer je $\Delta = \{x\in\mathbb{R}^n: \sum_{j\in V}x_j = 1 \text{ in } x_j \geq 0 \text{ za vsak } j\in V\}$ standardni simpleks iz $\mathbb{R}^n$.\\ \newline
Pravimo, da je x rešitev zgornjega problema če obstaja soseščina x-a $U\subseteq \Delta$ za katero je $f(x) > f(z)$ za vsak $z \in U\setminus\{x\}$. Podpora $\sigma(x)$ za $x\in\Delta$ je definirana kot indeksna množica pozitivnih komponenta vektorja x, torej $\sigma(x) = \{i\in V : x_i>0\}$.

\begin{definition}{Otežen vektor} \\
Za neprazno podmnožico C množice V lahko definiramo otežen vektor $x^C\in\Delta$, če ima množica C pozitivno skupno težo W(C). V tem primeru je
$$x^C_i=\begin{cases}
\frac{W_C(i)}{W(C)}& \text{; če $i \in C$},\\
0& \text{; sicer}.
\end{cases}$$
Za dominantno množico lahko torej vedno definiramo otežen vektor.
\end{definition}
\begin{theorem}
Če je C dominantna množica A, potem je njen otežen vektor $x^C$ rešitev zgornjega problema. Obratno, če je x* rešitev zgornjega problema, potem je njegova podpora $\sigma = \sigma(x^*)$ dominantna množica od A pri pogoju, da je $W_{\sigma\cup\{i\}}(i) \not= 0$ za vse $i \notin \sigma$.
\end{theorem}

\section{Povezava s teorijo grafov}
Naj bo $G=(V,E)$ neusmerjen graf, kjer je $V={1,2,\ldots,n}$ množica vozlišč in $E\subseteq V\times V$ množica povezav v grafu. Dve vozlišči $u, v \in V$ sta sosednji, če $(u, v) \in E$. Podmnožici vozlišč $C \subseteq V$ pravimo klika, če so si vsa vozlišča iz te množice med seboj sosednja.\\
Klika C na neusmerjenem grafu F je največja (maximal), če ne obstaja klika D na grafu G, tako da $C \subseteq D$ in $C \not= D$. Kliko C imenujemo maksimalna (maximum) klika, če ne obstaja klika na grafu G, ki bi vsebovala več vozlišč kot največja klika C. Število vozlišč v maksimalni kliki imenujemo klično število (clique number) in ga označimo z $\omega(G)$. \\
Matrika sosednosti grafa G je kvadratna matrika $A_G$, kjer je $(A_G)_{i,j}=1$, če $(i,j)\in E$, sicer pa $(A_G)_{i,j}=0$.\\
Na matriko sosednosti v neusmrejnem grafu lahko gledamo kot na matriko podobnosti v problemu razvrščanja in posledično lahko uporabimo dominantno množico da najdemo združbe znotraj grafa.\\
Glede na povezavo z teorijo optimizacije, upoštevamo naslednji kvadratični program
\begin{gather}
max f_\alpha(x) = x^T(A_G + \alpha I)x \\
\text{p. p. } x\in\Delta \subset \mathbb{R}^n
\end{gather}
Kjer je I identična matrika, $\alpha$ realno število in $\Delta$ simpleks.
\begin{theorem}
Naj bo graf G neusmerjen z matriko sosednosti $A_G$ in naj bo $0 < \alpha < 1$. Vsaka največja klika C grafa G je dominantna množica od $A_\alpha = A_G + \alpha I$. Obratno, če je C dominantna množica od $A_\alpha$ potem je C največja klika v G.
\end{theorem}

\section{Razvrščanje z uporabo dmoinantnih množic}
Naivna strategijabi lahko bilo oštevilčenje vseh podmnožic $C\subseteq V$ preverjanje pogojev iz Definicije 1. Ta rešitev je očitno precej neučinkovita, zato si bomo ogledali dve alternativni strategiji. Obe rešitvi izvirata iz teorije iger.

\subsection{Dinamika ponovitev}
The Replicator Dynamics (RD) are deterministic game dynamics
that have been developed in evolutionary game theory. This theory
originated in the early 1970s as an attempt to apply the principles
and tools of game theory to biological contexts, with a view to
model the evolution of animal, as opposed to human, behavior
(see the classical work by Maynard Smith, 1982 who pioneered
the field). It considers an idealized scenario whereby individuals
are repeatedly drawn at random from a large, ideally infinite,
population to play a two-player game. In contrast to classical
game theory, here players are not supposed to behave rationally
or to have complete knowledge of the details of the game. They
act instead according to an inherited behavioral pattern, or pure
strategy, and it is supposed that some evolutionary selection
process operates over time on the distribution of behaviors. A
general class of evolution equations is given by the following set
of ordinary differential equations ( Weibull, 1997 ):
x˙ i = x i g i ( x ) (7)
for i = 1 . . . n, where a dot signifies derivative with respect to time
and g = (g 1 , . . . , g n ) is a function with open domain containing .
Here, the function g i specifies the rate at which pure strategy i
replicates. It is usually required that the growth function g is regular
( Weibull, 1997 ), which means that it is Lipschitz continuous
and that g( x ) 	 x = 0 for all x ∈ . The former condition guarantees
that the system of the differential equation (7) has a unique
solution through any initial population state. The latter condition,
instead, ensures that the simplex  is invariant under (7) , namely,
any trajectory starting in  will remain in .
A point x is said to be a stationary (or equilibrium) point for
our dynamical systems, if x˙ i = 0 ( i ∈ S ). A stationary point x is
(Lyapunov) stable if for every neighborhood U of x there exists a
neighborhood V of x such that x (0) ∈ V implies x ( t ) ∈ U for all
t ≥ 0. A stationary point is said to be asymptotically stable if any
trajectory starting in its vicinity will converge to it as t → ∞ .
Payoff-monotonic game dynamics represent a wide class of
regular selection dynamics for which useful properties hold. Intuitively,
for a payoff-monotonic dynamics the strategies associated
to higher payoffs will increase at a higher rate. Formally, a regular
selection dynamics (7) is said to be payoff-monotonic if
g i ( x ) > g j ( x ) ⇔ (A x ) i > (A x ) j
for all x ∈  and i , j ∈ V .
Although this class contains many different dynamics, it turns
out that they share a lot of common properties. To begin, they
all have the same set of stationary points. Indeed, x ∈  is a
stationary point under any payoff monotonic dynamics if and only
if (A x ) i = x 	 A x holds for all i ∈ σ( x ) ( Weibull, 1997 ).
A well-known subclass of payoff-monotonic game dynamics is
given by
x˙ i = x i

φ( (A x ) i ) −

j∈ V
x j φ

(A x ) j
	


,
where φ( u ) is an increasing function of u . These models arise in
modeling the evolution of behavior by way of imitation processes,
where players are occasionally given the opportunity to change
their own strategies ( Weibull, 1997 ).
When φ is the identity function, that is, φ(u ) = u, we obtain
the standard continuous-time replicator equations ,
x˙ i = x i

(A x ) i − x
	 A x

, (8)
whose basic idea is that the average rate of increase x˙ i /x i equals
the difference between the average fitness of strategy i and the
mean fitness over the entire population.
Another popular model arises when φ(u ) = e ku , where k is
a positive constant. As k tends to 0, the orbits of this dynamics
approach those of the standard, first-order replicator model,
slowed down by the factor k ; moreover, for large values of k , the
model approximates the so-called best-reply dynamics ( Hofbauer
& Sigmund, 1998 ).
The replicator dynamics, and more in general any payoff
monotonic dynamics, have the following properties ( Hofbauer &
Sigmund, 2003; Weibull, 1997 ):
Theorem 4. Under any payoff monotonic dynamics the following
holds true:
• a Nash equilibrium is a stationary point;
• a strict Nash equilibrium is asymptotically stable;
• a stationary point x ∗ that is the limit of an interior orbit, i.e. , such
that σ ( x (t)) = V for all t ≥ 0 and lim t→∞ x (t ) = x ∗
, is a Nash
equilibrium
• a stable stationary point is a Nash equilibrium;
• an ESS is asymptotically stable.
In general, the converse of the implications in Theorem 4 do
not hold.
Furthermore, if we restrict our focus to symmetric payoff
matrices, i.e. , A = A 	 , then stronger properties hold, as stated in
the following theorem.
Theorem 5. If A = A 	 then the following holds:
• x 	 A x is strictly increasing along any non-constant trajectory of
(8) . In other words, for all t ≥ 0 we have d
dt [ x (t ) 	 A x (t )] > 0 ,
unless x is a stationary point. Furthermore, any such trajectory
converges to a (unique) stationary point;
• x is asymptotically stable if and only if x is an ESS.
In order to implement the continuous-time replicator dynamics
one can resort to some iterative method like, e.g. , the Runge-Kutta
method, to find an approximate solution to the ordinary differential
equations. Alternatively, one can adopt the discrete-time
counterpart of (8) , known as discrete-time replicator dynamics,
which are given by
x i (t + 1) = x i (t )
(A x (t)) i
x (t ) 	 A x (t )
,
for i ∈ V . Note that the discrete-time replicator dynamics are
simplex-invariant if A is nonnegative. This, however, is not a limitation
because any payoff A preserves its equilibria by a constant
shift. Hence, if A has negative entries, one can build a matrix
A = A + μE with positive entries by a proper choice of μ > 0,
where E is a matrix of all ones.
Since ESSs are asymptotically stable under the replicator
dynamics according to Theorems 4 and 5 , we can employ this
dynamics for the extraction of dominant sets, which are indeed in
correspondence to ESSs as described in Section 2.4 . Moreover, if
we assume symmetric payoff functions, we recover again the link
to optimization theory shown in Section 2.2 , for Theorem 5 implies
that the replicator dynamics locally maximize x 	 A x over the
simplex. Finally, motivated by the link to graph theory described
in Section 2.3 , replicator dynamics have also been employed as
heuristics for the maximum clique problem ( Pelillo & Torsello,
2006; Wu & Hao, 2015 ).
3.2. Infection and immunization dynamics
The Infection and Immunization Dynamics (InImDyn) are a class
of discrete-time dynamics that have been introduced in Rota Bulò
and Bomze (2011) to overcome some computational problems
afflicting standard evolutionary dynamics, including the replicator
dynamics. To mention a few, evolutionary dynamics like replicator
dynamics have a quadratic space/time complexity per iteration,
not all stationary points are Nash equilibria, they converge only
in the limit of infinitely-many iterations, and the detection of
the support of an equilibrium is cumbersome if the dynamics is
stopped before proper convergence ( i.e. , after finitely-many steps).
The InImDyn dynamics take the following form:
x (t + 1) = δS( x ) ( x )[ S( x ) − x ] + x , (9)
where we wrote x for x ( t ). The function S :  →  is a strategy
selection function that satisfies the following property:
S( x ) =

y for some y ∈ ϒ( x ) if ϒ( x )  = ∅ ,
x otherwise ,
(10)
where Y( x ) represents the set of so-called infective strategies for x
defined as
ϒ( x ) = { y ∈  : ( y − x )
	 A x > 0 } .
As long as there exists y ∈ Y( x ), the strategy x cannot be an equilibrium
by definition of infective strategy. If this is the case, the
dynamics (9) blends x and y (infects x with y ) until the violation
of the Nash condition caused by y vanishes, which happens by
considering a mixing factor δy ( x ) given by
δy ( x ) =

min


( x −y ) 	 A x
( y −x ) 	 A ( y −x ) , 1

, if ( y − x ) 	 A ( y − x ) < 0
1 , otherwise .
(11)
The InImDyn dynamics yields a fixed point when ϒ( x ) = ∅ . If
this is the case, x is also a Nash equilibrium:
Theorem 6. Let x ∈  be a strategy. Then the following statements
are equivalent:
(a) ϒ( x ) = ∅ : there is no infective strategy for x ;
(b) x is a Nash strategy;
(c) x is a fixed point under dynamics (9) .
Proof. See Rota Bulò and Bomze (2011) . 
Akin to the replicator dynamics, if we restrict to symmetric
payoff matrices, we have that the average payoff is strictly
increasing along any non-constant trajectory of InImDyn:
Theorem 7. Let { x ( t )} t ≥ 0 be a trajectory of (9) . Then for all t ≥ 0,
u ( x (t + 1) , x (t + 1) ) ≥ u ( x (t ) , x (t ) ) ,
with equality if and only if x (t ) = x (t + 1) , provided that the payoff
matrix is symmetric.
Proof. See Rota Bulò and Bomze (2011) . 
Depending on the choice of S( x ) in (9) , we may obtain
different dynamics. In the rest of the section, we review one
in particular, which is simple and leads to nice properties. Let
u ∈ arg max j ∈ V (A x ) j and v ∈ arg min j ∈ σ ( x ) (A x ) j and let
z
+ = e u and z
− =
x v
1 − x v
( x − e v ) + x ,
where e u is a null vector except for the u th element being 1. The
pure strategy selection function S Pure is defined as
S Pure ( x ) =

z + if ( z + − z −) 	 A x > 0
z − if ( z + − z −) 	 A x < 0
x otherwise .
Since the search space for an infective strategy is reduced from
 to a finite set, it is not obvious that S Pure ( x ) is a welldefined
selection function, i.e. , it satisfies (10) . However, one can
prove ( Rota Bulò & Bomze, 2011 ) thatthere exists an infective
strategy for x if and only if S Pure ( x ) is infective for x .
Another property that holds for Pure InImDyn, which is shared
also by the replicator dynamics, is the characterization of ESS
equilibria in terms of asymptotically stable points of the dynamics
under symmetric payoff matrices.
Theorem 8. A state x is asymptotically stable for InImDyn with S Pure
as strategy selection function if and only if x is an ESS, provided that
the payoff matrix is symmetric.
Proof. See Rota Bulò and Bomze (2011) . 
This selection function exhibits the nice property of rendering
the complexity per iteration linear in both space and time,
as opposed to the replicator dynamics, which have quadratic
space/time complexity per iteration ( Rota Bulò, Pelillo, & Bomze,
2011 ). We refer to Rota Bulò and Bomze (2011) and Rota Bulò et al.
(2011) for further analyses of the theoretical and computational
properties of InImDyn, among which convergence in finitely-many
steps and support separation in finite-time. We also refer the
interested reader to Ullrich (2016) for a recent development of a
continuous-time version of InImDyn.
3.3. Finding multiple clusters
The Replicator and InImDyn dynamics that we presented in the
previous sections allow to find upon convergence a single dominant
set. In principle, one aims at enumerating all dominant sets for the
clustering problem instance at hand, however, this is in general
computationally intractable because there might be exponentially
many clusters (see Section 2.5 ). In practice, having an incomplete,
but good, coverage of dominant sets is typically sufficient for application
purposes. In this section we review some computational
strategies for the (partial) enumeration of dominant sets.
Multi-start strategy. The first strategy is very naive and consists
in re-starting the dynamics from different, random points in
the simplex. This strategy is effective if the data consists of few
clusters with large basins of attraction, such that the probability of
sampling a point belonging to each basin is relatively high. Clearly,
there is no guarantee that this procedure will not extract the same
clusters multiple times. Hence, a post-processing step is devoted
to identifying and prune repeated clusters. The multi-start strategy
is optimal in the long-run, because it will eventually enumerate
all dominant sets, but it might require a very large number of
samples, thus being inefficient in practice.
Peeling-off strategy. This strategy follows a different philosophy
and finds application mainly in cases where one is not interested
in finding overlapping clusters. The idea is again very simple and
consists in iteratively extracting a dominant set and removing its
vertices. By doing so, there is the obvious guarantee that no dominant
set will be extracted multiple times. However, this strategy
is not optimal because on one hand the number of dominant
sets that can be potentially enumerated is upper-bounded by the
total number of vertices; on the other hand, except for the first
dominant set, all the following ones are not necessarily dominant
sets of the original clustering problem instance, for every time
vertices of a dominant set are peeled off, the underlying clustering
problem instance changes. Despite the lack of optimality, the
peeling-off strategy is computationally appealing and it has been
used in practice in many applications.
Destabilization strategy. This strategy has been proposed
in Torsello et al. (2008) and Rota Bulò, Torsello, and Pelillo
(2009) and it is based on the replicator dynamics. We will sketch
the method in intuitive terms here, while we refer to the original
paper for more details. The goal is to retain an optimal solution as
in the multi-start strategy, but, at the same time, to avoid getting
trapped by the same basins of attraction multiple times. The idea
is to iteratively render the dominant sets that are extracted over
time (and only those) unstable to the replicator dynamics, in order
to prevent their extraction at the succeeding iterations. In order
to destabilize a dominant set C , it is sufficient to add a new vertex
to the original graph, connected in a way that the characteristic
vector x C violates the ESS condition, while the set of non-extracted
ESSs remains unchanged (see, Rota Bulò et al., 2009; Torsello et al.,
2008 for details). Despite the theoretical validity of the approach,
there are some practical issues with this method due to the possible
emergence of periodic orbits that thwart the extraction of new
dominant sets. Moreover, the computational complexity grows
quadratically (assuming the number of RD iterations constant)
with the number of extracted dominant sets.
Tabu-list strategy. The last strategy has been proposed
in Kontschieder, Rota Bulò, Donoser, Pelillo, and Bischof (2012) and
it is based on the InImDyn dynamics. Again, we will simply sketch
the general idea and we refer to the original paper for more
details. The tabu-list strategy pursues the goal of an incomplete
enumeration of dominant sets akin to the peeling-off strategy,
yielding nevertheless a good coverage and without sacrificing the
optimality of the solution, i.e. , it guarantees that the extracted
clusters are dominant sets of the original clustering problem
instance. The idea is to keep a tabu-list during the clustering
process, i.e. , a list of vertices that should not be taken into account
during the dominant set extraction. In some sense, this is what
also the peeling-off strategy is doing, for all vertices of alreadyextracted
dominant sets are added to the tabu-list. However, in
order to prevent the possibility of extracting invalid dominant
sets, the tabu-list strategy checks the validity of putative dominant
sets found under the tabu-list constraint. In the specific, at each
dominant set extraction iteration, a dominant set C ⊂V is found by
running InImDyn with a random initialization under the constraint
that no vertex in the tabu-list T ⊂V can be used, i.e. , C ⊆ V T . This
constraint prevents C from being a valid dominant set in general,
i.e. , C might not be a dominant set with respect to the full vertex
set V . To examine the validity of the putative dominant set C , the
InImDyn dynamics is run again, initialized to the characteristic
vector x C and allowed to use all the vertices (including those in
the tabu-list). Let C
 ⊆ V be the dominant set found, which is
valid, but might not be novel. There are then 2 possible scenarios,
in which C
 can be regarded as a new, valid dominant set to be
added to the final clustering result: 1) C
 ⊆ T , 2) C
 ⊆ T , but C
 was
never found before. In the first case, the tabu-list is updated by
adding any vertex in C
 T ( e.g. , a random one), while in the second
case the tabu-list is kept as it is. If, instead, it turns out that C

⊆ T , but C
 has already been found, we reject C
 and add to the
tabu-list any vertex in V T . The tabu-list strategy guarantees that
either a valid and novel dominant set is extracted at each iteration,
or the tabu-list is increased by one, thus limiting the number of
iterations to max (| V |, # of clusters found). Note that one could also
add more vertices at time to the tabu-list to speed-up the clustering
process at the cost of eventually finding a smaller number of
dominant sets, as is actually done in Kontschieder et al. (2012) .

\end{document}