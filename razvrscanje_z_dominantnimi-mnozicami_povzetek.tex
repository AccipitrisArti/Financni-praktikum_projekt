\documentclass[a4paper]{article}
\usepackage[slovene]{babel}
\usepackage[utf8]{inputenc}
\usepackage[T1]{fontenc}
\usepackage{marvosym}
\usepackage{amssymb,amsmath}
\title{Razvrščanje z dominantnimi množicami - povzetek}
\author{Taja Debeljak, Anže Marinko \\ Finančni praktikum \\ Finančna matematika, Fakulteta za matematiko in fiziko}
\date{Jesen 2017}

\newtheorem{definition}{Definicija}[section]
\newtheorem{theorem}{Izrek}

\begin{document}
\maketitle
\section{Uvod}
Grupiranje (ang. Clustering) je postopek razvrščanja predmetov znotraj razreda v podrazrede (cluster) tako, da so si predmeti znotraj istega podrazreda bolj podobni med sabo, kot so si podobni z elementi iz ostalih podrazredov. \\
Problem združevanja lahko opišemo z uteženim grafom, ki ga definiramo kot trojico $G = (V,E,\omega)$, kjer je $V = {1,\ldots,n}$ končna množica vozlišč, $E \subseteq V \times V$ množica usmerjenih
povezav in $\omega: E \rightarrow \mathbb{R}$ funkcija, ki vsakemu vozlišču dodeli neko vrednost(težo). Vozlišča grafa G ustrezajo predmetom, ki jih je potrebno združevati. \\
Povezave predstavljajo, kateri predmeti so med seboj povezani, utežene povezave pa odražajo podobnosti med povezanimi predmeti. Poleg tega matrika $A_{i,j} = \omega(i,j) \text{ za vse } i, j \in V$ predstavlja podobnost med vozlišči. Imenujemo jo matrika podobnosti. \\
Osnovni lastnosti, ki morata zadostovati gruči, sta:
\begin{itemize}
\item Notranja homogenost: elementi, ki pripadajo gruči si morajo biti med seboj podobni
\item Maksimalnost: gruče ne moremo dodatno razširiti z uvedbo zunanjih elementov
\end{itemize}

\begin{definition}
Naj graf G predstavlja primer združevanja množic in naj bo $C \subseteq V$ neprazna podmnožica. \textit{Povprečna utežena vhodna stopnja} glede na C je definirana kot
$$awindeg_C(i) = \frac{1}{\lvert C \rvert}\sum_{j\in C}A_{i,j}$$
kjer $\lvert C \rvert$ predstavlja velikost množice C. Za $j\in C$ definiramo
$$\phi_C(i,j) = A_{i,j} - awindeg_C(j)$$
Funkcija $\phi_C(i,j)$ je \textit{mera relativne podobnosti} elementa i z elementom j glede na povprečno povezanost elementa i z elementi iz C.\\
\textit{Težo elementa} i glede na množico C definiramo kot
$$W_C(i)=\begin{cases}
1& \text{; če $\lvert C \rvert = 1$},\\
\sum_{j\in C\setminus{i}}\phi_{C\{i\}}(i,j)W_{C\setminus\{i\}}(j)& \text{; sicer}.
\end{cases}$$
Vrednost $W_C(i)$ nam pove koliko podpore prejme element i od elementov $C\setminus\{i\}$ glede na skupno podobnost z elementi iz $C\setminus\{i\}$. Pozitivne vrednosti nam povedo da je i močno
koleriran z $C\setminus\{i\}$.\\
\textit{Skupna teža množice} C pa je definirana z
$$W(C) = \sum_{i\in C}W_C(i)$$
\end{definition}
\begin{definition}{Dominantna množica}\\
Neprazni množici $C \subseteq V$ za katero je $W(T) > 0$ za vsako neprazno množico $T \subseteq C$ pravimo dominantna množica, če velja:
\begin{enumerate}
\item $W_C(i) > 0$ za vse $i \in C$
\item $W_{C\cup\{i\}}(i) < 0$ za vse $i \notin C$
\end{enumerate}
\end{definition}

\section{Povezava s teorijo optimizacije}
Če se omejimo na simetrične povezanosti, torej A je simetrična matrika, potem lahko dominantno množico zapišemo kot rešitev naslednjega standardnega kvadratičnega programa
\begin{gather}
max f(x) = x^TAx \\
\text{p. p. } x\in\Delta \subset \mathbb{R}^n
\end{gather}
Kjer je $\Delta = \{x\in\mathbb{R}^n: \sum_{j\in V}x_j = 1 \text{ in } x_j \geq 0 \text{ za vsak } j\in V\}$ standardni simpleks iz $\mathbb{R}^n$.\\ \newline
Pravimo, da je x rešitev zgornjega problema če obstaja soseščina x-a $U\subseteq \Delta$ za katero je $f(x) > f(z)$ za vsak $z \in U\setminus\{x\}$. Podpora $\sigma(x)$ za $x\in\Delta$ je definirana kot indeksna množica pozitivnih komponenta vektorja x, torej $\sigma(x) = \{i\in V : x_i>0\}$.

\begin{definition}{Otežen vektor} \\
Za neprazno podmnožico C množice V lahko definiramo otežen vektor $x^C\in\Delta$, če ima množica C pozitivno skupno težo W(C). V tem primeru je
$$x^C_i=\begin{cases}
\frac{W_C(i)}{W(C)}& \text{; če $i \in C$},\\
0& \text{; sicer}.
\end{cases}$$
Za dominantno množico lahko torej vedno definiramo otežen vektor.
\end{definition}
\begin{theorem}
Če je C dominantna množica A, potem je njen otežen vektor $x^C$ rešitev zgornjega problema. Obratno, če je x* rešitev zgornjega problema, potem je njegova podpora $\sigma = \sigma(x^*)$ dominantna množica od A pri pogoju, da je $W_{\sigma\cup\{i\}}(i) \not= 0$ za vse $i \notin \sigma$.
\end{theorem}

\section{Povezava s teorijo grafov}
Naj bo $G=(V,E)$ neusmerjen graf, kjer je $V={1,2,\ldots,n}$ množica vozlišč in $E\subseteq V\times V$ množica povezav v grafu. Dve vozlišči $u, v \in V$ sta sosednji, če $(u, v) \in E$. Podmnožici vozlišč $C \subseteq V$ pravimo klika, če so si vsa vozlišča iz te množice med seboj sosednja.\\
Klika C na neusmerjenem grafu F je največja (maximal), če ne obstaja klika D na grafu G, tako da $C \subseteq D$ in $C \not= D$. Kliko C imenujemo maksimalna (maximum) klika, če ne obstaja klika na grafu G, ki bi vsebovala več vozlišč kot največja klika C. Število vozlišč v maksimalni kliki imenujemo klično število (clique number) in ga označimo z $\omega(G)$. \\
Matrika sosednosti grafa G je kvadratna matrika $A_G$, kjer je $(A_G)_{i,j}=1$, če $(i,j)\in E$, sicer pa $(A_G)_{i,j}=0$.\\
Na matriko sosednosti v neusmrejnem grafu lahko gledamo kot na matriko podobnosti v problemu razvrščanja in posledično lahko uporabimo dominantno množico da najdemo združbe znotraj grafa.\\
Glede na povezavo z teorijo optimizacije, upoštevamo naslednji kvadratični program
\begin{gather}
max f_\alpha(x) = x^T(A_G + \alpha I)x \\
\text{p. p. } x\in\Delta \subset \mathbb{R}^n
\end{gather}
Kjer je I identična matrika, $\alpha$ realno število in $\Delta$ simpleks.
\begin{theorem}
Naj bo graf G neusmerjen z matriko sosednosti $A_G$ in naj bo $0 < \alpha < 1$. Vsaka največja klika C grafa G je dominantna množica od $A_\alpha = A_G + \alpha I$. Obratno, če je C dominantna množica od $A_\alpha$ potem je C največja klika v G.
\end{theorem}

\end{document}